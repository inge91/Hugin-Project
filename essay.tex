%%%%%%%%%%%%%%%%%%%%%%%%%%%%%%%%%%%%%%%%%
% Simple Sectioned Essay Template
% LaTeX Template
%
% This template has been downloaded from:
% http://www.latextemplates.com
%
% Note:
% The \lipsum[#] commands throughout this template generate dummy text
% to fill the template out. These commands should all be removed when 
% writing essay content.
%
%%%%%%%%%%%%%%%%%%%%%%%%%%%%%%%%%%%%%%%%%

%----------------------------------------------------------------------------------------
%	PACKAGES AND OTHER DOCUMENT CONFIGURATIONS
%----------------------------------------------------------------------------------------

\documentclass[12pt]{article} % Default font size is 12pt, it can be changed here

\usepackage{geometry} % Required to change the page size to A4
\geometry{a4paper} % Set the page size to be A4 as opposed to the default US Letter

\usepackage{graphicx} % Required for including pictures

\usepackage{float} % Allows putting an [H] in \begin{figure} to specify the exact location of the figure
\usepackage{wrapfig} % Allows in-line images such as the example fish picture

\usepackage{lipsum} % Used for inserting dummy 'Lorem ipsum' text into the template

\linespread{1.2} % Line spacing

%\setlength\parindent{0pt} % Uncomment to remove all indentation from paragraphs

\graphicspath{{./Pictures/}} % Specifies the directory where pictures are stored

\begin{document}

%----------------------------------------------------------------------------------------
%	TITLE PAGE
%----------------------------------------------------------------------------------------

\begin{titlepage}

\newcommand{\HRule}{\rule{\linewidth}{0.5mm}} % Defines a new command for the horizontal lines, change thickness here

\center % Center everything on the page

\textsc{\LARGE University of Amsterdam}\\[1.5cm] % Name of your university/college

\HRule \\[0.4cm]
{ \huge \bfseries The Hugin Project}\\[0.4cm] % Title of your document
\HRule \\[1.5cm]

\begin{minipage}{0.4\textwidth}
\begin{flushleft} \large
\emph{Author:}\\
Inge \textsc{Becht} % Your name
\end{flushleft}
\end{minipage}

{\large \today}\\[3cm] % Date, change the \today to a set date if you want to be precise

%\includegraphics{Logo}\\[1cm] % Include a department/university logo - this will require the graphicx package

\vfill % Fill the rest of the page with whitespace

\end{titlepage}

%----------------------------------------------------------------------------------------
%	TABLE OF CONTENTS
%----------------------------------------------------------------------------------------

\tableofcontents % Include a table of contents

\newpage % Begins the essay on a new page instead of on the same page as the table of contents 

%----------------------------------------------------------------------------------------
%	INTRODUCTION
%----------------------------------------------------------------------------------------

\section{Introduction} % Major section

When talking about reasoning with uncertainty an important part of it consists
of how to represent random variables that are dependable of other events when finding out
their likelihood. The Rule of Bayes shows this idea, but can be extended to the
bayesian model in which all events that are causally related are visualised with
their probabilities.

In this essay a bayesian network is constructed to show the basic idea of its
inner woking and to
show its prediction capabilities of a simplified problem (with only around 10
nodes). The subject of the
bayesian network consits of the question \emph{Will i be moving to a new
appartment?}. For the probabilitie assignment of the different events, a 2 year span is
kept in mind. When assigning the values the noisy OR model (A way to assign
probabilitie values in a condition probability table) is used when possible
and else an explanation is given why it wouldn't make sense to use it. 
At the end the probability of me moving in the next two year are given with some
possible explanation for the outcome and some
experimentation with prior knowledge will show some more possible outcomes.

%------------------------------------------------

\section{The idea of a Bayesian network} % Sub-section

A Bayesian network is a model consisting of different nodes in which each one
has a causal relation towards other nodes that it has a connection with. The
direction of the arrow shows the causal relation between two points.

A bayesian network has the following retraints in its construction:
\begin{itemize}
    \item The model should be acyclic. 
    \item All events should have a condition probability table 
\end{itemize}



%------------------------------------------------

\section{Creating the network} % Sub-section


The first step in constructing the network was to think about all kind of events
that are direct causes of moving to a new appartment. The following elements
came to mind directly:

\begin{itemize}
    \item Money fluctuation
        \begin{itemize}
            \item Earning more
            \item Earning less
            \item Earning the same
        \end{itemize}
    \item Noisy neighbours
    \item House becomes unlivable
\end{itemize}

The probability of these events were hereafter added towards

%------------------------------------------------

\subsection{Occuring problems} % Sub-sub-section



%------------------------------------------------

\subsubsection{ Adding The money to the network} 


%----------------------------------------------------------------------------------------
%	MAJOR SECTION 1
%----------------------------------------------------------------------------------------

\section{Content Section} % Major section


%------------------------------------------------

\subsection{Subsection 1} % Sub-section

\subsubsection{Subsubsection 1} % Sub-sub-section


%------------------------------------------------

\subsubsection{Subsubsection 2} % Sub-sub-section


%------------------------------------------------

\subsubsection{Subsubsection 3} % Sub-sub-section





%----------------------------------------------------------------------------------------
%	MAJOR SECTION X - TEMPLATE - UNCOMMENT AND FILL IN
%----------------------------------------------------------------------------------------

%\section{Content Section}

%\subsection{Subsection 1} % Sub-section

% Content

%------------------------------------------------

%\subsection{Subsection 2} % Sub-section

% Content

%----------------------------------------------------------------------------------------
%	CONCLUSION
%----------------------------------------------------------------------------------------

\section{Conclusion} % Major section


%----------------------------------------------------------------------------------------
%	BIBLIOGRAPHY
%----------------------------------------------------------------------------------------

\begin{thebibliography}{99} % Bibliography - this is intentionally simple in this template

\bibitem[Figueredo and Wolf, 2009]{Figueredo:2009dg}
Figueredo, A.~J. and Wolf, P. S.~A. (2009).
\newblock Assortative pairing and life history strategy - a cross-cultural
  study.
\newblock {\em Human Nature}, 20:317--330.
 
\end{thebibliography}

%----------------------------------------------------------------------------------------

\end{document}
