%%%%%%%%%%%%%%%%%%%%%%%%%%%%%%%%%%%%%%%%%

% Simple Sectioned Essay Template
% LaTeX Template
%
% This template has been downloaded from:
% http://www.latextemplates.com
%
% Note:
% The \lipsum[#] commands throughout this template generate dummy text
% to fill the template out. These commands should all be removed when 
% writing essay content.
%
%%%%%%%%%%%%%%%%%%%%%%%%%%%%%%%%%%%%%%%%%

%----------------------------------------------------------------------------------------
%	PACKAGES AND OTHER DOCUMENT CONFIGURATIONS
%----------------------------------------------------------------------------------------

\documentclass[12pt]{article} % Default font size is 12pt, it can be changed here

\usepackage{geometry} % Required to change the page size to A4
\usepackage{subfigure}
\usepackage{placeins}
\geometry{a4paper} % Set the page size to be A4 as opposed to the default US Letter

\usepackage{graphicx} % Required for including pictures

\usepackage{float} % Allows putting an [H] in \begin{figure} to specify the exact location of the figure
\usepackage{wrapfig} % Allows in-line images such as the example fish picture

\usepackage{lipsum} % Used for inserting dummy 'Lorem ipsum' text into the template

\linespread{1.2} % Line spacing

%\setlength\parindent{0pt} % Uncomment to remove all indentation from paragraphs

\graphicspath{{./Pictures/}} % Specifies the directory where pictures are stored

\begin{document}

\begin{enumerate}
    \item  The idea of common knowledge is important because it is able to
        extend the idea of collective knowledge(where everybody knows $\phi$) with recursion, so that
        everybody knows that \dots  everybody knows $\phi$.
        This creates a new way of reasoning within a group, as more information
        is available to each individual about what is known.
        In the paper it is stated that this idea of common knowledge makes
        driving on the road possible(because everybody has to know that
        everybody else knows the rule to even dare to drive).
        Another possible application is in case of negotiation. If I know that
        you know how much an item  is worth, and you know that, it seems both parties
        will get a fair deal. 
    \item proposition p never seems to be common knowledge as in case u
        
    \item 
        \begin{equation}
        (M,s) \models C_G\varphi \Leftrightarrow (M,s) \models E_G(\varphi \wedge
        C_G\phi) 
        \end{equation}
        Can be rewritten as:
        \begin{equation}
            (M,s) \models C_G\varphi \Leftrightarrow (M,s) \models E_G(\varphi) \wedge
            E_G (C_G\varphi) 
        \end{equation}
        Because of the definition on page 43 this can be rewritten as:
        \begin{equation}
            (M,s) \models C_G\varphi \Leftrightarrow (M,s) \models E_G(\varphi) \wedge
            E_G(E_G^{k-1}\varphi) \mbox{ for } k = 2,3, ... 
        \end{equation}
        When we set $(M,s)\models C_G\varphi$ to true this can be rewritten as:
        \begin{equation}
            (M,s) \models E_G^k\varphi  \mbox{ for } k = 1,2 ... \Leftrightarrow (M,s) \models E_G(\varphi) \wedge
            E_G(E_G^{k-1}\varphi) \mbox{ for } k = 2, 3, ... 
        \end{equation}
        More rewriting:
        \begin{equation}
            (M,s) \models E_G^k\varphi  \mbox{ for } k = (1,2 ... \Leftrightarrow (M,s) \models E_G(\varphi) \wedge
            E_G^{k}\varphi 	\mbox{ for } k = 1, 2, ...  
        \end{equation}
        If $E_G^k$ is true then $E_G(\varphi)$ as well, so that:
        \begin{equation}
            (M,s) \models E_G^k\varphi  \mbox{ for } k = 1, 2, ... \Leftrightarrow (M,s)
            \models
            E_G^{k}\varphi \mbox{ for } k = 1, 2...  
        \end{equation}

        In case $(M,s) \models C_G\varphi$ is not set to true the derivation
        could not be made, hence the case would be that $(M,s) \models
        \lnot E_G(\varphi \wedge C_G\varphi$
    \item
        For an attack to happen not only does the second commander have to receive the
        messenger, but also does the first commander need to know the second
        commander knows that he wants to plan the attack and agrees to it. The
        second commander than nees to know that de first commandeer knows this
        and so on. Common knowledge in this case would enable the attack
        happening as then both parties would know that the other party new
        indefintely.
    \item \emph{Synchronous} in this problem means that all nonfaulty(not
        lying) generals after the same amount of time choose for the same time
        to attack\\
        \emph{Searching for agreement} is the process for the nonfaulty
        processors to agree upon one and the same strategy, while the fault
        processors in \emph{Byzantine failure} mode try to confsue these faulty
        processesors without their knowledge.\\
        \emph{Point to point} means that all generals are fully connected to all
        other generals, and so is the messaging.
    \item In the paper is stated that always SBA can me made, but in the case of \emph{Byzantine failure}
        mode for faulty processors there have to be at least $n = 3*f + 1$ more
        nonfaulty processors than faulty ones (where n is nonfaulty and f is
        faulty), else no agreement can be made. This is only true when
        signatures can be forgeable(if a faulty processor can seem to be an
        unfaulty one), else SBA can be made with arbitrary many
        faulty processors. The SBA can be made in f+1 rounds, where f is the
        upper bound of faulty processors in the system.
         
        

\end{enumerate}
\end{document}

