%%%%%%%%%%%%%%%%%%%%%%%%%%%%%%%%%%%%%%%%%

% Simple Sectioned Essay Template
% LaTeX Template
%
% This template has been downloaded from:
% http://www.latextemplates.com
%
% Note:
% The \lipsum[#] commands throughout this template generate dummy text
% to fill the template out. These commands should all be removed when 
% writing essay content.
%
%%%%%%%%%%%%%%%%%%%%%%%%%%%%%%%%%%%%%%%%%

%----------------------------------------------------------------------------------------
%	PACKAGES AND OTHER DOCUMENT CONFIGURATIONS
%----------------------------------------------------------------------------------------

\documentclass[12pt]{article} % Default font size is 12pt, it can be changed here

\usepackage{geometry} % Required to change the page size to A4
\usepackage{subfigure}
\usepackage{placeins}
\geometry{a4paper} % Set the page size to be A4 as opposed to the default US Letter

\usepackage{graphicx} % Required for including pictures

\usepackage{float} % Allows putting an [H] in \begin{figure} to specify the exact location of the figure
\usepackage{amssymb} % Allows putting an [H] in \begin{figure} to specify the exact location of the figure
\usepackage{wrapfig} % Allows in-line images such as the example fish picture

\usepackage{lipsum} % Used for inserting dummy 'Lorem ipsum' text into the template

\linespread{1.2} % Line spacing

%\setlength\parindent{0pt} % Uncomment to remove all indentation from paragraphs

\graphicspath{{./Pictures/}} % Specifies the directory where pictures are stored

\begin{document}

\begin{enumerate}
    \item  The idea of common knowledge is important because it is able to
        extend the idea of collective knowledge(where everybody knows $\phi$) with recursion, so that
        everybody knows that \dots  everybody knows $\phi$.
        This creates a new way of reasoning within a group, as more information
        is available to each individual about what is known. In particular it
        helps with simultaneous decision making.
        \\
        In the paper it is stated that this idea of common knowledge makes
        driving on the road possible(because everybody has to know that
        everybody else knows the rule to even dare to drive).
        Another possible application is in case of modeling negotiation. If I know that
        you know how much an item  is worth, and you know that, it seems both parties
        will simultaneous get to a fair price.
    \item For proposition p to be common knowledge it has to be so that given a
        state x, $(M, x) \models C_G p$ where $G = {Agent 1, Agent 2}$. So both
        need to know that p is the case in a state x and both have to know that
        both know that this is the case etc. (as can be derived from the
        definitions of page 43)\\
        In state $t$, $p$ is not common knowledge as $agent 1$ considers both
        $t$ and $s$
        possible (he cannot see the difference between both). Although he knows
        that $agent 2$ knows, this is not sufficient.\\
        In state $s$, $p$ is no common knowledge either, as shown with the
        reasoning on page 42 (again $agent 1$ cannot differentiate between s and
        t, which both have a different value for p).\\
        In state $u$, $agent 2$ considers both $s$ and $t$ possible, but since in both
        worlds $p$ is true, he knows $p$. $Agent 1$ only considers $u$ possible so
        he knows $p$ is true as well. They also both know that about eachother,
        so in this state there is common knowledge about P.
        \\
        \\
        Unfortunately, I do not understand the tautology question, so I will not
        be able to answer that.
        
    \item 
        \begin{equation}
        (M,s) \models C_G\varphi \Leftrightarrow (M,s) \models E_G(\varphi \wedge
        C_G\phi) 
        \end{equation}
        Can be rewritten as:
        \begin{equation}
            (M,s) \models C_G\varphi \Leftrightarrow (M,s) \models E_G(\varphi) \wedge
            E_G (C_G\varphi) 
        \end{equation}
        Because of the definition on page 43 this can be rewritten as:
        \begin{equation}
            (M,s) \models C_G\varphi \Leftrightarrow (M,s) \models E_G(\varphi) \wedge
            E_G(E_G^{k-1}\varphi) \mbox{ for } k = 2,3, ... 
        \end{equation}
        When we set $(M,s)\models C_G\varphi$ to true this can be rewritten as:
        \begin{equation}
            (M,s) \models E_G^k\varphi  \mbox{ for } k = 1,2 ... \Leftrightarrow (M,s) \models E_G(\varphi) \wedge
            E_G(E_G^{k-1}\varphi) \mbox{ for } k = 2, 3, ... 
        \end{equation}
        More rewriting:
        \begin{equation}
            (M,s) \models E_G^k\varphi  \mbox{ for } k = (1,2 ... \Leftrightarrow (M,s) \models E_G(\varphi) \wedge
            E_G^{k}\varphi 	\mbox{ for } k = 1, 2, ...  
        \end{equation}
        If $E_G^k$ is true then $E_G(\varphi)$ as well, so that:
        \begin{equation}
            (M,s) \models E_G^k\varphi  \mbox{ for } k = 1, 2, ... \Leftrightarrow (M,s)
            \models
            E_G^{k}\varphi \mbox{ for } k = 1, 2...  
        \end{equation}

        In case $(M,s) \models C_G\varphi$ is not set to true the derivation
        could not be made, hence the case would be that $(M,s) \nvDash
        E_G(\varphi \wedge C_G\varphi)$
    \item
        For an attack to happen not only does the second commander have to receive the
        messenger, but also does the first commander need to know the second
        commander knows that he wants to plan the attack and agrees to it. The
        second commander than needs to know that de first commandeer knows this
        and so on. Common knowledge in this case would enable the attack
        happening as then both parties would know that the other party new
        indefintely.
    \item \emph{Synchronous} in this problem means that all nonfaulty(not
        lying) generals after the same amount of time choose for the same time
        to attack\\
        \emph{Searching for agreement} is the process for the nonfaulty
        processors to agree upon one and the same strategy, while the fault
        processors in \emph{Byzantine failure} mode try to confsue these faulty
        processesors without their knowledge.\\
        \emph{Point to point} means that all generals are fully connected to all
        other generals, and so is the messaging.
    \item In the paper is stated that always SBA can me made, but in the case of \emph{Byzantine failure}
        mode for faulty processors there have to be at least $n = 3*f + 1$ more
        nonfaulty processors than faulty ones (where n is nonfaulty and f is
        faulty), else no agreement can be made. This is only true when
        signatures can be forgeable(if a faulty processor can seem to be an
        unfaulty one), else SBA can be made with arbitrary many
        faulty processors. The SBA can be made in f+1 rounds, where f is the
        upper bound of faulty processors in the system.
    \item From theorem 6.1 we can deduce that $(R, r, t) \models \varphi$ means
        that at a possible time $t$ of run $r$ $\varphi$ follows. Now
         $(R, r, t) \models  C_N\varphi$ means that at time t from run r it is
         common knowledge for all members of group N that $\varphi$ is true.
         This is if and only if every member of N knows that $\varphi$ and knows
         that $\varphi$ is common knowledge (as is shown in exercise 3).
    \item So in time t of run r $\varphi$ is common knowledge of group N, or in
        other words, all members of N know that all members of N know that all
        members of N know... etc. that $\varphi$. A possible example, that might
        not be the most interesting I admit is when a group of N people pick a
        card from the deck and shows it to the rest of the group. The card
        distribution is now common knowledge at a given interval in this
        specific run (the next time interval the cards might be turned down and
        shuffeled). The group of N people that show their card can be extended
        with "faulty people" who keep their card hidden and thus shows to the
        rest which are faulty and which are not.
    \item 
        \begin{itemize}
            \item A \emph{knowledge-based protocol} is a protocol wherein the
                processors explicitly act on the knowledge they have. The next
                action is solely based on that.
            \item A \emph{Crash failure} occurs when a faulty processor no
                longer transmits its messages, and thus does no longer
                communicate with the rest of the processors. In the traditional Byzantine
                agreement task this could be messengers that were intercepted
                and killed.\\
                An \emph{Ommission failure} occurs when a faulty processor only
                some of the time does not deliver its messages to other
                processors. In the traditional task this could be lazy
                messengers who do not take their jobs seriously.
            \item \emph{Byzantine Errors} occurs when faulty processors are not
                following protocol (or crash failure or omission failure occurs),
                like the lying generals in the traditional
                task. This is extra hard because there is not an abscence of
                information, but wrong information spread around in the system,
                and when forgery is possible it takes longer for the nonfaulty
                systems to find agreement about their value.
        \end{itemize}
    \item I'll use $p_i$ for processors on place $i \in {1, 2, 3, 4}$
        In case of crash failure:\\
        For 0000:
        First round:\\
        $p_1$ sends data to all other processors\\
        $p_2$ sends data to all other processors\\
        $p_3$ fails and does not send data\\
        $p_4$ sends data to all other processors\\
        
        After first round system halts and all nonfaulty processors decide on 0

        For 1111:
        First round:\\
        $p_1$ sends data to all other processors\\
        $p_2$ sends data to all other processors\\
        $p_3$ fails and does not send data\\
        $p_4$ sends data to all other processors\\
        
        After first round system halts and all nonfaulty processors decide on 1


    \item
        Halpern  states that 
        \emph{If we restrict our
            attention to crash failures or omission
            failures, there are only finitely many
            possible global states at any time, so that wecan calculate the truth of a
            fact at any point simply by applying the definition of $\models$ given in Section 2}.
        I find this a rather unhelpful explanation as I have no idea what a
        $\models$ definition is, and nowhere in section 2 is is called like
        that.
        I assume that a \emph{standard} protocol correspond to the basic
        definition as we learned in the multi agent slides, where a protocol
        determines an action for each state.\\
\end{enumerate}
\end{document}

